\chapter{Einleitung}
%Ziel dieser Bachelorarbeit ist es die QR-Zerlegung zu beschreiben. 
Als QR-Zerlegung bezeichnet man die Zerlegung der Matrix $A$ in eine  orthogonale Matrix $Q$ und eine obere Dreiecksmatrix $R$.
\begin{align*}
	A = Q \cdot R
\end{align*} 

In dieser Arbeit wird die Berechnung der QR-Zerlegung mittels Householder-Transfomation betrachtet. 

Neben der Householder-Transfomation kann die QR-Zerlegung auch mittels Givens-Rotationen oder mit dem Gram-Schmidtschen Orthogonalisierungsverfahrens berechnet werden.

Die QR-Zerlegung bietet folgende Möglichkeiten

Mit eine QR-Zerlegung kann man:
\begin{itemize}
	\item Gleichungssysteme mit schlechter Kondition lassen sich mittels QR-Zerlegung stabiler als mit dem Gausschen Eliminationsverfahren (LR-Zerlegung) lösen. 

	\item Lösen von linearen Ausgleichsproblemen durch die Methode der kleinsten Fehlerquadrate.
	
	\item Im QR-Verfahren ist die QR-Zerlegung eine Kernoperation, weil in jedem Iterationsschritt eine QR-Zerlegung berechnet wird.
	Das QR-Verfahren berechnet die Eigenwerte einer Matrix.	
\end{itemize}

Weil die QR-Zerlegung bei diesen Problemen oft verwendet wird, empfiehlt es sich, die Berechnung der OR-Zerlegung zu optimieren.



Um die Prozessorkapazität optimal zu nutzen, ist es sinnvoll die QR-Zerlegung mit einem Cache optimierten Algorithmus zu berechnen.


\section{Intel MKL (Math Kernel Library)}

Die Intel MKL ist eine Bibliothek, in der mathematische Funktionen implementiert sind. Darunter sind auch BLAS-Routinen (Basic Linear Algebra Subprogramms) und LAPACK-Routinen (Linear Algebra Package) \cite{DGEQR2}.



MKL ist eine Abkürzung für Math Kernel Library 

Kapitel über die Wichtigkeit der Intel MKL.

