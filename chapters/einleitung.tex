\chapter{Einleitung}
%Ziel dieser Bachelorarbeit ist es die QR-Zerlegung zu beschreiben. 
Als QR-Zerlegung versteht man die Zerlegung der Matrix $A$ in eine  orthogonale Matrix $Q$ und eine obere Dreiecksmatrix $R$.
\begin{align*}
	A = Q \cdot R
\end{align*} 

In dieser Arbeit wird nur auf die Berechnung der QR-Zerlegung mittels Householder-Transfomation eingegangen. 

Neben der Householder-Transfomation kann man die QR-Zerlegung auch noch mittels Givens-Rotationen oder mit dem Gram-Schmidtschen Orthogonalisierungsverfahrens berechnet werden.




\begin{itemize}
	\item Wozu dient die QR-Zerlegung?
	
	\item Besser beim Lösen von Gleichungssystemen mit schlechter Kondition.
	
	\item Lösen von l	inearen Ausgleichsproblemen
	
	\item QR-Verfahren (Eigenwerte)
	
	\item Warum muss die QR-Zerlegung schnell sein? zb Kern operation beim QR verfahren.
	
	\item Warum Cache-Optimiert? Um Rechner capazität optiaml aus zu nuzen.
\end{itemize}


\section{Intel MKL}

Die Intel MKL ist ein


MKL ist eine Abkürzung für Math Kernel Library 

Kapitel über die Wichtigkeit der Intel MKL.

