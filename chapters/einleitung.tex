\chapter{Einleitung}
%Ziel dieser Bachelorarbeit ist es die QR-Zerlegung zu beschreiben. 
Als QR-Zerlegung bezeichnet man die Zerlegung der Matrix $A$ in eine  orthogonale Matrix $Q$ und eine obere Dreiecksmatrix $R$.
\begin{align*}
	A = Q \cdot R
\end{align*} 

In dieser Arbeit wird die Berechnung der QR-Zerlegung mittels Householder-Transformation betrachtet. 

Neben der Householder-Transformation kann die QR-Zerlegung auch mittels Givens-Rotationen oder mit dem Gram-Schmidtschen Orthogonalisierungsverfahrens berechnet werden.

Die QR-Zerlegung bietet folgende Möglichkeiten:
%Mit eine QR-Zerlegung kann man:
\begin{itemize}
	\item Gleichungssysteme mit schlechter Kondition lassen sich mittels QR-Zerlegung stabiler als mit dem Gausschen Eliminationsverfahren (LR-Zerlegung) lösen. 

	\item Lösen von linearen Ausgleichsproblemen durch die Methode der kleinsten Fehlerquadrate. 
	
%	\item Im QR-Verfahren ist die QR-Zerlegung eine Kernoperation, weil in jedem Iterationsschritt eine QR-Zerlegung berechnet wird.	Das QR-Verfahren berechnet die Eigenwerte einer Matrix.	
	
	\item Berechnung von Eigenwerten einer Matrix mittels QR-Verfahren. Im QR-Verfahren ist die QR-Zerlegung eine Kernoperation, da in jedem Iterationsschritt eine QR-Zerlegung berechnet wird.
\end{itemize}

%Weil die QR-Zerlegung bei diesen Problemen oft verwendet wird, empfiehlt es sich, die Berechnung der OR-Zerlegung zu optimieren.

Weil die QR-Zerlegung bei den aufgezählten Problemen häufig verwendet wird, empfiehlt es sich, die Berechnung der OR-Zerlegung zu optimieren.

Um die Prozessorkapazität optimal zu nutzen, ist es sinnvoll, die QR-Zerlegung mit einem cache-optimierten Algorithmus zu berechnen.
Dadurch ist es möglich, die angegebene Maximalleistung des Prozessors zu erreichen.

\subsubsection{Cache-Optimierung}
Der Cache ist ein schneller Datenspeicher. Daten, die vom Prozessor verarbeitet werden sollen, müssen immer zuerst in den Cache geladen werden.
Der Zugriff auf Daten, die im RAM liegen, dauert sehr viel länger als der Zugriff auf Daten, die im Cache liegen.
Aus diesem Grund ist es sinnvoll, die Algorithmen so zu gestalten, dass eine optimale Übertragung erfolgt.

Ziele der Cache-Optimierung:
\begin{itemize}
	\item Mehrfaches Laden von Daten soll vermeiden werden.
	\item Die Daten müssen in der Reihenfolge, in der sie verwendet werden, im RAM liegen.
	\item Sequentiell im RAM liegende Daten werden durch Prefetching effizient in den Cache geladen.
\end{itemize}

Prefetching ist die Eigenschaft des Prozessors, Zugriffsmuster auf den Speicher zu erkennen und vorherzusagen.	

\subsubsection{Intel MKL}
Die Intel MKL (Math Kernel Library) \cite{mkl} ist eine Bibliothek der Firma Intel, in welcher mathematische Funktionen enthalten sind.
%Die Intel MKL ist eine Bibliothek der Firma Intel. Die Abkürzung MKL steht für Math Kernel Library.
%In der Bibliothek sind mathematische Funktionen enthalten.
Die MKL implementiert diese Funktionen sehr effizient, damit der Prozessor die angegebene theoretische Maximalleistung erreichen kann.

%Die MKL beinhaltet BLAS-Routinen (Basic Linear Algebra Subprogramms) und LAPACK-Routinen (Linear Algebra Package) \cite{DGEQR2}.

%Es sind Funktionen der linearen Algebra wie BLAS (Basic Linear Algebra Subprogramms) und LAPACK (Linear Algebra Package) implementiert.

In der MKL sind die beiden Programmbibliotheken BLAS (Basic Linear Algebra Subprogramms) und LAPACK (Linear Algebra Package) implementiert. 
Diese Funktionen aus der Teilmenge der linearen Algebra sind für die Berechnung der QR-Zerlegung elementar erforderlich.

BLAS enthält grundlegende und LAPACK weiterentwickelte Funktionen der linearen Algebra.

\subsubsection{Blas kapiltel motivieren}

Ziel der Arbeit ist es einen Cache-optimierten Algorithmus zu beschreiben und diesen auf die BLAS-Schnitstelle an zu passen.









