\chapter{Implementierung und Benchmarks}
Kurze Beschreibung der HPC-Bibliothek

Die verwendete Bibliothek wurde in der Vorlesung HPC1 entwickelt.



objektorientiert%\label{zur hpc vorlesung}

eventuell Beispiel

\section{Fehlerschätzer}

Es wurde der Fehlerschätzer von ATLAS verwendet.
\begin{align}
	err = \dfrac{\|A - QR\|_i}{\|A\|_i \cdot \min(m,n) \cdot \varepsilon}
\end{align}
mit $\|\cdot\|_i$ passender Norm und $\varepsilon$ die kleinste darstellbare Zahl.

Die QR-Zerlegung ist gut genug falls $ err < 1 $.

Als Norm wurde die Unendlichnorm $\|\cdot\|_\infty$ gewählt.
Die Unendlich-Norm entspricht der Zeilensummennorm, die für eine Matrix $A \in \mathbb{R}^{m\times n}$ gegeben ist durch
\begin{align*}
	\|A\|_\infty = \max_{i=1,...,m} \sum_{j=1}^{n} |a_{ij}|
\end{align*}

$\epsilon$ ist auf dem Test-System $2.220446\cdot10^{-16}$

\section{MKL-Wrapper}

\section{Benchmarks}

peak performance einzeichnen