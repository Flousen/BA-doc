\chapter{Implementierung und Benchmarks}
Die verwendete Bibliothek wurde in der Vorlesung High Performance Computing 1 entwickelt \cite{HPC1}.

Die Bibliothek ist in C++ geschrieben. Es sind Klassen für Matrizen und Vektoren implementiert, sowie einige BLAS-Routinen.

Die Matrix-Klassen erlauben den zugriff auf Matrixblöcke. 



eventuell Beispiel

\section{Fehlerschätzer}

Es wurde der Fehlerschätzer von ATLAS verwendet. Nach Quelle Fragen!
\begin{align}
	err = \dfrac{\|A - QR\|_i}{\|A\|_i \cdot \min(m,n) \cdot \varepsilon}
\end{align}
$\|\cdot\|_i$ ist eine passende Norm.
Die Matrizen $Q$ und $R$ sind die QR-Zerlegung der Matrix $A \in \mathbb{R}^{m \times n}$.
$\varepsilon$ ist die kleinste darstellbare Zahl.\\
Die QR-Zerlegung ist gut genug falls der Fehler kleiner 1 ist $ err < 1 $.

Als Norm wurde die Zeilensummennorm $\|\cdot\|_\infty$ gewählt.
Diese ist für eine Matrix $A \in \mathbb{R}^{m\times n}$ gegeben durch
\begin{align*}
	\|A\|_\infty = \max_{i=1,...,m} \sum_{j=1}^{n} |a_{ij}|
\end{align*}

$\epsilon$ ist auf dem Test-System $2.220446\cdot10^{-16}$

\section{MKL-Wrapper}



\section{Benchmarks}

peak performance einzeichnen

Taktrate * Register breite * 2 

Taktrate : 3.20GHz  
AVX-Register 256-Bit 4 double

25,6

	 