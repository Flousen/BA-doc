\chapter{Block Reflector}
Das Produkt aus Householder-Transformationen $H_1 \cdot ... \cdot H_n$ lässt sich schreiben als 
\begin{align*}
	H_1 \cdot ... \cdot H_n = I - VTV^T
\end{align*}
mit einer unteren Dreiecksmatrix $V \in \mathbb{R}^{m \times n}$ die die Housholder-Vektoren enthält und eine oberen Dreiecksmatrix $T \in \mathbb{R}^{n \times n}$ \cite{Joffrain:2006:AHT:1141885.1141886}\\
Beweis:\\
n=2
Vorwärts 
\begin{align*}
	H_1 H_2 x &= (I-\tau_1 v_1 v_1^T)(I-\tau_2 v_2 v_2^T)x\\
	= &(I - \tau_1 v_1 v_1^T - \tau_2 v_2 v_2^T +  \tau_1 v_1 v_1^T \tau_2 v_2 v_2^T )x\\
  = &x - \tau_1 v_1 v_1^T x - \tau_2 v_2 v_2^T x + \tau_1 \tau_2 v_1 (v_1^T v_2 )v_2^T x\\
  = &x - \tau_1 v_1 v_1^T x - \tau_2 v_2 v_2^T x + \tau_1 \tau_2 (v_1^T v_2 ) v_1 v_2^T x\\
\end{align*}
Rückwärts
\begin{align*}
H_{1,2} x &= (I - V T V^T) x = x - V T V^T x\\
&= x - (v_1, v_2)
\begin{pmatrix}
a & b \\ 0 & c
\end{pmatrix}
\begin{pmatrix}
v_1^T \\ v_2^T 
\end{pmatrix}
x\\
&= x - (v_1, v_2)
\begin{pmatrix}
a & b \\ 0 & c
\end{pmatrix}
\begin{pmatrix}
v_1^T x \\ v_2^T x
\end{pmatrix}\\
&= x - (v_1, v_2)
\begin{pmatrix}
a v_1^T x + b v_2^T x\\  c v_2^T x
\end{pmatrix}\\
&= x - v_1(a v_1^T x + b v_2^T x) - v_2 (c v_2^T x)\\
&= x - a v_1 v_1^T x - b v_1 v_2^T x - c v_2 v_2^T x
\end{align*}
Koeffizienten Vergleich
\begin{align*}
a &= \tau_1 \\
b &=-  \tau_1 \tau_2 (v_1^T v_2) \\
c &= \tau_2 \\
T &=
\begin{pmatrix}
\tau_1 & - \tau_1 \tau_2 (v_1^T v_2)\\ 0 & \tau_2
\end{pmatrix}
\end{align*}
\newpage
n=3\\
Vorwärts \\
\begin{align*}
	H_1 H_2 H_3 x &= (I-\tau_1 v_1 v_1^T)(I-\tau_2 v_2 v_2^T)(I-\tau_3 v_3 v_3^T)x\\
	= &(I - \tau_1 v_1 v_1^T - \tau_2 v_2 v_2^T +  \tau_1 v_1 v_1^T \tau_2 v_2 v_2^T )(I-\tau_3 v_3 v_3^T)x\\
	= &(I - \tau_1 v_1 v_1^T - \tau_2 v_2 v_2^T - \tau_3 v_3 v_3^T \\
	&+ \tau_1 v_1 v_1^T \tau_2 v_2 v_2^T 
	+ \tau_1 v_1 v_1^T \tau_3 v_3 v_3^T
	+ \tau_2 v_2 v_2^T \tau_3 v_3 v_3^T\\
	&- \tau_1 v_1 v_1^T \tau_2 v_2 v_2^T \tau_3 v_3 v_3^T )x \\
	= &x - \tau_1 v_1 v_1^Tx - \tau_2 v_2 v_2^Tx - \tau_3 v_3 v_3^Tx \\
    &+ \tau_1 \tau_2  (v_1^T v_2) v_1  v_2^T x
	+ \tau_1 \tau_3  (v_1^T v_3) v_1  v_3^T x
	+ \tau_2 \tau_3  (v_2^T v_3) v_2  v_3^T x\\
    &- \tau_1 \tau_2  \tau_3 (v_1^T v_2 v_2^T v_3) v_1  v_3^T x \\
\end{align*}
Rückwärts\\
\begin{align*}
  H_{1,2,3} x &= (I - V T V^T) x = x - V T V^T x\\
  &= x - (v_1, v_2, v_3)
  \begin{pmatrix}
    a & b & c\\ 
    0 & d & e\\
    0 & 0 & f
  \end{pmatrix}
  \begin{pmatrix}
    v_1^T \\ v_2^T \\ v_3^T
  \end{pmatrix}
  x\\
  &= x - (v_1, v_2, v_3)
  \begin{pmatrix}
    a & b & c\\ 
    0 & d & e\\
    0 & 0 & f
  \end{pmatrix}
  \begin{pmatrix}
    v_1^T x \\ v_2^T x \\ v_3^T
  \end{pmatrix}\\
  &= x - (v_1, v_2, v_3)
  \begin{pmatrix}
    a v_1^T x + b v_2^T x + c v_3^T\\ 
    d v_2^T x + e v_3^T \\
    f v_3^T
  \end{pmatrix}\\
  =& x - v_1(a v_1^T x + b v_2^T x + c v_3^T x) \\ 
   & - v_2 ( d v_2^T x + e v_3^T x) \\ 
   & - v_3 ( f v_3^T ) \\
  =& x - a v_1 v_1^T x - b v_1 v_2^T x - c v_1 v_3^T x \\
   & - d v_2 v_2^T x - e v_2 v_3^T \\
   & - f v_3 v_3^T
\end{align*}

Koeffizienten Vergleich
\begin{align*}
	a =&  \tau_1\\
	b =& -\tau_1 \tau_2 (v_1^T v_2 ) \\
	c =& -\tau_1 \tau_2  \tau_3 (v_1^T v_2 v_2^T v_3) + \tau_1 \tau_3  (v_1^T v_3)\\
	d =&  \tau_2 \\
	e =& -\tau_2 \tau_3  (v_2^T v_3)\\
    f =&  \tau_3\\
	T =&
	\begin{pmatrix}
	a & b & c\\ 
	0 & d & e\\
	0 & 0 & f
	\end{pmatrix} =
	\begin{pmatrix}
		\tau_1 & -\tau_1 \tau_2 (v_1^T v_2 ) & - \tau_1 \tau_2  \tau_3 (v_1^T v_2 v_2^T v_3) + \tau_1 \tau_3  (v_1^T v_3)\\ 
		0 & \tau_2 &  -\tau_2 \tau_3  (v_2^T v_3)\\
		0 & 0 & \tau_3
	\end{pmatrix}
\end{align*}
Mit Induktion kann man zeigen... siehe paper
Im Paper wird gezeit wie man das verallgemeinern kann. \cite{Joffrain:2006:AHT:1141885.1141886}
%\subsection{Orthogonal}
%Eine quadratische Matrix $Q \in \mathbb{R}^{n \times n}$ ist orthogonal, dann gilt
%\begin{align*}
%	QQ^T = Q^TQ = I
%\end{align*}
%Produkt orthogonaler Matrizen ist orthogonal. Sei $A^{-1} = A^T, B^{-1} = B^T$
%\begin{align*}
%	(AB)^{-1} = B^{-1}A^{-1} = B^TA^T = (AB)^T
%\end{align*}
%
%Die Househlder-Transformation $H=I - 2 \dfrac{vv^T}{v^Tv}$ ist symmetrisch und orthogonal das heißt $H^{-1} = H^T$\\
%Da $vv^T$ symmetrisch ist ($(vv^T)^T= vv^T$), folgt
%\begin{align*}
%	H^T = \left(I - 2 \dfrac{vv^T}{v^Tv} \right)^T = I - 2 \dfrac{vv^T}{v^Tv} = H
%\end{align*}
%Orthogonalität
%\begin{align*}
%	HH^T = \left(I - 2 \dfrac{vv^T}{v^Tv} \right)\left(I - 2 \dfrac{vv^T}{v^Tv} \right)
%	= I - 2 \dfrac{vv^T}{v^Tv} - 2 \dfrac{vv^T}{v^Tv} + \underbrace{ 4 \dfrac{vv^Tvv^T}{(v^Tv)^2}}_{=4 \dfrac{(v^Tv)vv^T}{(v^Tv)^2}=4 \dfrac{vv^T}{v^Tv}} = I
%\end{align*}
%\\
%$\Rightarrow H=I-VTV^T$ und $Q$ sind orthogonal