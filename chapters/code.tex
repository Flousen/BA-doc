\section{Implementierung}
Der Komplette Code steht in folgendem \textit{git repository} auf GitHub Verfügung: \\
\url{https://github.com/Flousen/Bachelorarbeit} 

Der folgende Code ist im Repository zu finden unter: /src/hpc/mklblas/qr.hpp\\
Die Datei \textit{qr.hpp} enthält folgende Funktionen:
\begin{itemize}
    \item \textit{householderVector}\\
    Die Funktion berechnet den Householder-Vektor.
	\item \textit{qr\_unblk} \\
	Die Funktion berechnet die QR-Zerlegung mittels Householder-Transformation, mit dem ungeblocketen Algorithmus.
	\item \textit{larft}\\
	Die Funktion berechnet die Matrix $T$ die zur Anwendung mehrerer Householder-Transformationen notwendig ist (siehe Abschnitt \ref{calcT}).
	\item \textit{larfb}\\
	Die Funktion wendet mehrere Householder-Transformationen auf eine Matrix an (siehe Abschnitt \ref{applyT}). 
	\item \textit{qr\_blk}\\
	Die Funktion berechnet die QR-Zerlegung mittels Householder-Transformation, mit dem geblockten cache-optimierten Algorithmus.
\end{itemize}

\vspace{.7cm}
\lstset{numbers=left,firstnumber=1}
\lstinputlisting{/home/flo/Bachelorarbeit/src/hpc/mklblas/qr.hpp}


